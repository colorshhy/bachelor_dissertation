\chapter{视觉表达分类器}

由于查询集数量十分庞大,我们想做一个视觉表达分类器自动地判定一个查询是否能视觉表达。我们对查询经过图片搜索,所得到的图片集,用图片集来判断一个查询是否能视觉表达。利用现有的TLC图像识别技术,将图片放入卷积神经网络中学习出它的特征特征向量。图像识别技术依据这个图片特征向量进行图像识别,调用特征提取接口将这个向量经过降维、归一化后将图片转化为一个256维的图片向量。

我们将查询集的图片向量来判断一个查询是否能视觉表达。我们以图片集的数量,图片向量间的欧拉平均距离以及它的标准差,余弦平均距离以及标准差,聚类后熵、最大类的数量、最大类的半径为作为分类特征,使用线性分类算法训练分类模型。我们开始直接使用短文本的人工标记作为训练数据,但是因为短文本的负例太多,我们后来部分添加了单词的人工标记,修正了正负例的比例。最后使用了882个查询作为训练数据,分类器参数如下表\ref{tab:svm_result}所示

\begin{table}[htbp]
\centering
\caption{视觉表达分类器} \label{tab:svm_result}
\begin{tabular}{ |c|c|c|c|}
    \hline
		正确率 & 精确率 & 回归率 &  F1\\
	\hline 
		73.94 & 75.56 & 51.52 & 61.26 \\
 	\hline
\end{tabular}
%\note{这里是表的注释}
\end{table}