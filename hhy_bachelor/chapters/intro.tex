\chapter{简介}
%\section{模板简介}
短文本与图像关联是当下热门且富有意义的研究方向。短文本分词研究已较为深入,卷积神经网络成熟运用使得图像识别有了成熟的商业产品,而短文本与图像的等价表达仍有待研究。随着大数据时代的来临,互联网多媒体资源丰富,文字与图像的关系更为紧密。这使得短文本的视觉表达成为可能。

%研究动力
本研究的动力来自于参与NTCIR-13\footnote{日本国立情报学研究所NII组织,是信息检索国际性评估会议}短文本对话STC-2\footnote{该比赛分为中文和日文两个部分,中文部分语料库来自微博,日文部分来自Tiwwer }比赛。比赛中文部分由日本国立情报学研究所NII和华为诺亚方舟实验室联合组织,比赛内容是给定一个微博(在本文中用Post表示)内容,构建一个检索或者生成系统来得到连贯的、有意义的回复(本文中用Cmt表示)。陈仲夏师兄在去年STC-1比赛的主要想法是通过TF-IDF和Word2Vec算法在语料库中寻找相似的微博,用相似的微博语料库中对应的评论作为此微博Post的评论Cmt。本研究主要贡献在寻找相似的微博内容的时候,在传统的文字相似性上添加视觉相似性来寻找和重新评价其关联性。

%论文想法来源
随着多媒体时代的到来,我们无论是在自己发微博写朋友圈的时候,还是浏览新闻看公众号文章都会配上相关的图片,甚至近些年流行的表情包都表明图片在信息交流传播中占据了重要的一部分。这一方面极大的充实了图片搜索引擎的内容,另一方面也说明了文字在再创造和演变的过程中将视觉、听觉等感官高度编码,变成一个同意的利于传播的格式。但是人在理解短文本的内容时往往会带有自身的经验和联想。例如,有一个微博说到“今天又老一岁了”,根据我们的常识我们能推理出发这个微博的博主今天过生日,因此“生日快乐!”是合适的回答。但是如今的计算机还并不具备常识,目前的主流的方法中它只能通过微博内容文字与文字之间的相似性来表示微博语义上的相似性。但是在短文本中的语义往往需要较多的背景知识,单从文字角度挖掘微博之间的相关性会存在局限性。我们想法是,通过过搜索引擎的图片搜索功能,使将微博的文字“解码”,扩充其视觉感官的部分,扩充计算机在理解微博语义部分的背景知识。


%短文本对话是指根据一段短文字,计算机模仿人类给出连贯的有意义的回复。过去文本都是从语料库中找寻合适的回答,   近些年随着机器学习的发展,循环神经网络RNN和长短周期神经网络LSTM的出现文字转向量Word2Vec技术的成熟,文本生成技术是的生成文本回复短对话有种有短文本对话是当下热门的研究方向,以微软小冰、苹果Safari为代表,各家互联网公司都在积极研发对话机器人。现在得到对话问答


