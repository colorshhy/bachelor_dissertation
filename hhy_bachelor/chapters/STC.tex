
\chapter{短文本对话Short Text Conversation}
人与电脑之间的自然语言交流是最具挑战性的AI问题之一,涉及语义理解,推理和常识知识的运用。尽管过去几十年来对人机交互研究工作进行了大量的努力,但令人遗憾的是,这个问题的进展非常有限。其中一个主要原因是缺乏大量的真实对话数据。

在短文本对话任务中,我们只考虑一个简化版本的问题:由两个短文组成的一轮对话,前者是用户的初始帖子,后者是计算机给出的评论。我们把它称为短文本对话(STC)。由于在Twitter和微博等社交媒体上提供了大量短文本对话数据,我们预计在使用大数据的问题研究中可以取得重大进展,就像在机器翻译,社区问答等领域一样。随着社交媒体的出现和移动设备的广泛传播,通过短消息的对话已成为重要的沟通方式。许多现实中应用可以从短文本对话的研究中受益,例如手机上的自动消息回复,Siri等语音助手以及用于智能家居设备上各种聊天机器人。

在NTCIR-12上短文本对话的作为试点任务提出,让有兴趣的自然语言对话的研究人员聚在一起。在NTCIR-12,短文本对话(STC)被认为是一个信息检索(IR)问题,通过在日语子任务中保持一个大量的Twitter中的子博客Twitter和Twitter Twitter的留言对,然后找到一个聪明的方法来重用这些现有的评论来回应新的帖子。在中文子任务中,语料库来自于微博。

在今年我参加的NTCIR-13中,除了基于检索的方法之外,主办方还考虑了基于生成的方法来生成“新”评论。基于生成的方法已经成为一个热点研究课题,近年来受到最多的关注,而基于检索的方法是否完全被替代或者与基于生成的短文本对话(STC)任务的方法组合在一起仍然是一个开放的问题。NTCIR-13的短文本对话任务提供了一个透明的平台,通过进行综合评估来比较基于检索和基于新生成方法。此外,主办方鼓励参与者探索一些有效的方式来结合两种方法来获得更智能的聊天机器人。

短文本对话(STC)的一个简单方法,也许是大部分人第一种想要尝试的方法,就是将其作为信息检索(IR)问题:维护一个大型的短文对话数据库,并开发主要基于信息检索(IR)技术的会话系统。给定一个初始帖子(Post)A,系统搜索语料库并返回最合适的评论(Cmt)。存储库中的评论(Cmt)最初是针对Post A以外的一些帖子发布的,但我们假设语料库足够大,包含了所有可能存在的帖子(Post),因此我们假设可以将其重新用于对A的合理评论(Cmt)。也就是说,我们处理更简单的基于检索的短文本对话(STC),而不是追求基于生成的短文本对话(STC),即从用户的初始帖子(Post)生成适当的评论(Cmt)。利用先进的信息检索(IR)技术和大数据,即使是基于检索的短文本对话(STC)系统也可能在每一轮会话中最终都会像人类一样表现出来。

因此,我们想研究的关键问题是:给定一个新的帖子(Post),如何搜索语料库来返回适当的(即类似人的)评论?陈仲夏师兄,在NTCIR-12上给出了他的想法:假设语料库

在我的论文研究中,主要针对信息检索的方法的前期工作,研究短文本的视觉表达提供了一种新的检索语料库的
%短文本对话是指根据一段短文字,计算机模仿人类给出连贯的有意义的回复。过去文本都是从语料库中找寻合适的回答,近些年随着机器学习的发展,循环神经网络RNN和长短周期神经网络LSTM的出现文字转向量Word2Vec技术的成熟,文本生成技术是的生成文本回复短对话有种有短文本对话是当下热门的研究方向,以微软小冰、苹果Safari为代表,各家互联网公司都在积极研发对话机器人。现在得到对话问答




