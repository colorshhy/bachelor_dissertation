\chapter{词的视觉表达}

文字在演变过程中是将我们看到的感受到的记录下来,在最早的文字中我们用圆圈来代表太阳,然后经过演变发展成我们今天的“日”。说到视觉表达,我们第一直觉反应就是我们看见的物体形状。所以我们先研究短文本短的极限,单词,我们直觉上他们是短文本中最能视觉表达的部分。在我们的常识中,一个句子越长它的含义越丰富,因而越难寻找到一个图像准确的表达它的语义。为了最后能训练更好的短文本视觉表达自动分类器,我们希望找到在更短的文本中找到视觉表达的词。这章我们来讨论短文本的特例,词的视觉表达。

一个句子的语义成分或者说是关键成分在人脑海中有一些固定概念,相反一些辅助成分我们对其没有概念,它们只是帮助我们理解语义。因此,我们大胆推测在句子中起辅助作用的成分在计算机中也没有视觉表达。换句话说,这些成分的视觉表达很大概率在对话系统中不起作用。例如,我们对“?”进行图片检索,搜索的图片大部分都是问号形状的图片,但是这个图像表达对我们帖子间的关联没有意义。\cite{song2009identification}因此,词性可以帮助我们筛去一些大概率没有视觉表达的部分。我们只选择名词和形容词作为我们的实验对象。

\subsection{中文常用词}
我们使用了商务印书馆2008年出版的图书《现代汉语常用词表》中的56,008个现代词语以及频率。对其进行词性标注以后筛选了其中的名词后还有23,742个词。我们选出了其中300个进行了标记,记为常用词组。

\subsection{诗集中的词}
我们收集了1920年以前的519位诗人的诗词,并将其利用结巴分词进行分词。将诗句分词后进行词频统计和词性标注。词性标注后,名词有11,771个,形容词有1,390个,我们抽选了最高频的150个名词和50个形容词作为高频组,随机选取的150个名词和50个形容词作为随机组,两个组合并称为诗集组。对400个词进行图片搜索后组成问卷调查,进行如图\ref{fig:quest_vision}所示的人工标记。

\begin{table}[htbp]
\centering
\caption{诗词视觉表达调查问卷的人工标记一致度(Fleiss'Kappa)} \label{tab:poem_label_kappa}
\begin{tabular}{|c|c|c|c|}
    \hline
		计算方法 & 诗集组 &  高频组 &  随机组  \\
	\hline
		A,B,C 	&  0.345198 &  0.346088 &  0.337107 \\
	\hline
		A,(BC)  &  0.421180 & 	0.534748 & 0.335548 \\
	\hline
		(AB),C  &  0.400946 & 0.381584 &  0.409502	\\
	\hline
		(AC),B  &  0.240052 & 0.222053 & 0.252251	\\
 	\hline
\end{tabular}
%\note{这里是表的注释}
\end{table}

因为本文的研究利用到了大量的人工标记,为了人工标记的可信度,我们使用Fleiss'Kappa算法来测量这些人工标记的一致度。算法统计了标记人员在每个小题上的选择数量,然后综合每个小题得到所以标注人员之间的一致度。卡帕值在[-1,1],卡帕值越高说明人们的选择越一致,卡帕值越低说明他们对问题的理解存在分歧,当所有人在所有的问题都选了同一个选项时,卡帕值为1,相反如果卡帕值小于0说明标注员之间的对这个问题的分歧不如随机选择。

如表\ref{tab:poem_label_kappa}所示,表中的数值对应标注员的卡帕值。计算方法中A、B、C对应问卷调查中的问题“Query(查询)是否能够用图片视觉表达”中的选项。A:不能视觉表达;B:部分能视觉表达;C:能视觉表达。在没有合并的算法(A,B,C)中,我们欣喜的看到他们都取得了比较高的一致度。说明我们的标注结果是可靠的。我们试图确定人们对能视觉表达一致度更高,还是对不能视觉表达一致度更高。我们将三个选项两两合并,将AC合并的选项是没有道理的因此它的卡帕值下降是合理的,另外两个选项交不合并的算法都有明显提高,说明这种合并在选择一致度上是合理的。虽然高频组和低频组对合并算法各有偏好,在所有组中,在A,(BC)算法中取得了更高的一致度,它表示人们对不能图像表达有更高的一致性,倾向于混淆部分能表达(B)和完全能表达(C)。我们在训练视觉表达分类器时,也会用到这种算法。

\subsection{小结}

\begin{table}[htbp]
\centering
\caption{词视觉表达调查问卷结果} \label{tab:poem_result}
\begin{tabular}{ |c|c|c|c|c|}
    \hline
		查询集 & 常用词组 & 诗集组 &  高频组 & 随机组 \\
	\hline 
		Wquery得分 & 0.683333 & 0.77625   & 0.815 & 0.7376 \\
 	\hline
\end{tabular}
%\note{这里是表的注释}
\end{table}

在表\ref{tab:poem_result}中记录了标注结果。我们在调查问卷中的问题“Query(查询)是否能够用图片视觉表达”中的选项。A:不能视觉表达标记为0分;B:部分能视觉表达标记为1分;C:能视觉表达标记为2分。归一化后算得如上表中的得分。在高频的词汇中我们能看到在高频的词汇中能够有大部分的单词可以被判定为视觉表达。现代汉语常用词组和诗集组的比较中,我们能够看到,诗集中的常用词更能够视觉表达。这可能与诗在用词的时候更考虑意象,喜欢用“太阳”“世界”“朋友”这样的词汇,而常用词更多的是“问题”“关系”“社会”这样的单词。在高频词和随机组的比较中,我们发现高频词和低频词在视觉表达上存在差距,我们认为这部分原因来自于搜索引擎,如果一个短语出现的频率很少,那么网上对它的资料更倾向于杂乱无章。还有一部分也符合我们的认知,如果一个东西的频率越低,我们对其越没有清晰的印象。






