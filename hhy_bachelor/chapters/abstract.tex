\begin{abstract}
短文本囿于字数极少,这使得人在理解时会联系到各种感官经验,这其中就包括视觉。比如“猴子爬树”,我们在理解时会联想到树木的图像。本课题旨在利用搜索引擎,对一段短文本能否使用图像表达进行探究。本课题试图解决以下几个问题:第一,定义短文本的视觉表达;第二,通过调查问卷,人工标记具体短文本是否能视觉表达;第三,通过监督学习的方法,自动识别能视觉表达的短文本。
	

\keywords{近似搜索\zhspace{} 图片搜索\zhspace{} 可视化\zhspace{}询问分类}
\end{abstract}

\begin{enabstract}
The short text is limited by the few words, which makes it possible for people to understand the sensory experience, which includes vision. For example, "monkeys climb trees", we will understand the tree when the image. This topic aims to use the search engine, a short text can use the image expression to explore. This question attempts to solve the following questions: First, the definition of short text of the visual expression; Second, through the questionnaire, manually marked the specific short text can be visual expression; Third, through the supervision of learning methods, automatic identification of visual expression The short text.


\enkeywords{Similarity Search, Image Retrieval, Visualization, query classification}
\end{enabstract}
